\fancyhead[R]{Lecture 4}
\begin{example}
    For example, take $K\varphi:=\int K(s,t)\varphi(s)ds$, the HS class is given by \begin{equation*}
        \begin{split}
            \sum_{j=1}\langle K\varepsilon_j,K\varepsilon_j \rangle&=\sum_j \int K\varepsilon_j(x)K\varepsilon_j(x)dx\\
            &= \sum_j \int \pa{\int K(y,x)\varepsilon_j(y)dy}^2dx\\
            &\overbrace{=}^{?}\sum_j \int \int K(y,x)^2dydx < \infty
        \end{split}
    \end{equation*}
\end{example}

% Similarly, we define the \textbf{trace class} of operators as
% \begin{definition}
%     K is in the trace class if \begin{equation*}
%         \sum_{j=1}^{\infty} \lambda_j < \infty
%     \end{equation*}
% \end{definition}
% \begin{remark}
%     $\sum_{j=1}^{\infty} \lambda_j = \tr(K)=\sum_{j=1}\langle K\varepsilon_j,\varepsilon_j \rangle$ for any $\{\varepsilon_j\}$ an orthonormal basis in $\cale$.
% \end{remark}
\begin{example}
    Let us take \begin{equation*}
        \begin{split}
            K\varphi&=\E\bra{\varphi(Y)|Z }\\&=\int \varphi(y)f(y|z)dy \\
            &= \int \varphi(y)\frac{f(y,z)}{f(z)}\in L_Z^2\\
        \end{split}
    \end{equation*} $K$ is an operator from $L_Y^2 \to L_Z^2$ where $Y$ and $Z$ are random variables with joint density function $f(y,z)$.
    Then, by definition the adjoint operator $K^*$ is given by $K^*\psi(y)=\int \psi(z)f(z|y)dz$
    The Hilbert-Schmidt class is given by \begin{equation*}
        \begin{split}
            \int \pa{\int \varphi_i(y)\frac{f(y,z)}{f(z)}dy}^2dz&= \int \int \varphi_i(y)\frac{f(y,z)}{f(z)}\varphi_i(y)\frac{f(y,z)}{f(y)}dy dz\\&= \int \int \frac{f(y,z)}{f(z)}\frac{f(y,z)}{f(y)}dy dz
        \end{split}
    \end{equation*}
    On a side note, $$\frac{f(y,z)}{f(y)f(z)}=\sum_{j=1}^{\infty} \lambda_j \varphi_j(y)\psi_j(z)$$ {\color{red} (why?)} and $\sum_{j=1}\lambda_j^2$ is a measure of the dependence between $Y$ and $Z$.
\end{example}
Convergence in Hilbert-Schmidt class (?)

% Recall how to the scalar product in function space:
% \begin{equation*}
%     \angs{\varphi(s),\psi(s)}=\int \varphi(s)\psi(s) d\mu(s)
% \end{equation*}
% where $\mu(s)$ is some measure (?)

\subsection{Linear Equations}
For a linear equation $K\varphi=r$, we define two types of linear operators
\begin{enumerate}
    \item Type 1: $K$ is compact
    \item Type 2: $K=I-T$ where $T$ is compact
\end{enumerate}
{\color{red} (Why two types?)}
\subsubsection{Genralized Inverse}
The problem \begin{equation*}
    \min_{\varphi\in\cale}{\norm{K\varphi-r}^2}
\end{equation*} has a solution $\varphi_0$ if and only if $r\in \calr\pa{K}+\calr K(r)^{\perp}$.
Question: Is the column and null space (kernel) perpendicular to each other?
In finite dimension, $\F=\calr\pa{K}+\calr K(r)^{\perp}$ while infinite dimension, $\F=\overline{\calr\pa{K}}+\calr K(r)^{\perp}$.
If $\varphi_0$ exists, then $\varphi_0+\varphi$ for any $\varphi \in \caln\pa{K}$ is also a solution. We want to find the minimum norm generalized inverse.
% We denote the solution to the problem as $\varphi^+$ where $K\varphi^+=r$. 
We perform a SVD of the operator $K$ and get
$\lambda_i,\set{\varphi_i},\set{\psi_i}$. Thus, $r$ can be expressed in terms
of the orthonormal basis defined by $\set{\psi_i}$ as $$r=\sum_i
    \lambda_i\angs{r,\psi_i}\psi_i.$$ Similarly for $\varphi=\sum_i
    \lambda_i\angs{\varphi,\varphi_i}\varphi_i$. The linear equation becomes \begin{equation*}
    K\varphi=\sum_i \lambda_i\angs{\varphi,\varphi_i}K\varphi_i=\sum_i \lambda_i\angs{\varphi,\varphi_i}\psi_i=\sum_i \lambda_i\angs{r,\psi_i}\psi_i=r
\end{equation*}
which implies that \begin{align*}
     & \lambda_i\angs{\varphi,\varphi_i}=\angs{r,\psi_i}           \\
     & \angs{\varphi,\varphi_i}= \frac{\angs{r,\psi_i}}{\lambda_i}
\end{align*} for $\lambda_i\neq 0$.
\begin{definition}[Generalized inverse]
    If $\sum_i \frac{\angs{r,\psi_i}}{\lambda_i}^2 < \infty$, then the solution to the linear equation is given by \begin{equation*}
        \varphi^+=\sum_i \frac{\angs{r,\psi_i}}{\lambda_i}\varphi_i
    \end{equation*}
\end{definition}
\begin{remark}
    We can see that $\varphi^+$ is not continuous w.r.t. $r$ as $\lambda_i$ can go to zero.
\end{remark}
\subsubsection{Regularized solution}
When $\lambda_i$ goes to zero, we have a problem. We can regularize the
solution by add a regularization term.
\begin{definition}[Tikhonov regularization]
    The new problem is \begin{equation*}
        \min  \norm{K\varphi-r}^2+\alpha^2\norm{\varphi}^2, \alpha>0
    \end{equation*}
    The regularized solution is given by \begin{equation*}
        \varphi_\alpha=\sum_i \frac{\lambda_i}{\lambda_i^2+\alpha^2}\angs{r,\psi_i}\varphi_i
    \end{equation*}
\end{definition}
\begin{proof}
    We make use of gâteau derivative.
\end{proof}
\begin{remark}
    This is similar to what we do in OLS regression where $\hat{\beta}= (X^\top X)^{-1}X^\top y$ yet $X^\top X$ is not invertible. We deal with the issue using \textbf{Ridge regression} by $\hat{\beta}=(X^\top X+\alpha I)^{-1}X^\top y$.
\end{remark}
We now prove the existence of the solution:
\begin{enumerate}
    \item $\alpha I+K^*K$ is a compact operator from $\cale$ to $\cale$ and it satisfies $(\alpha I+K^*K)\varphi_j=(\alpha+\lambda_j^2)\varphi_j$.
    \item $K^*r=\sum_i \lambda_i\angs{r,\psi_i}\varphi_i$.
\end{enumerate}
As a result, \begin{equation}\label{eq:varphi_alpha}
    \varphi_\alpha=(\alpha I+K^*K)^{-1}K^*r=\sum_i \frac{\lambda_i}{\lambda_i^2+\alpha^2}\angs{r,\psi_i}\varphi_i
\end{equation}
The behavior of $\varphi_\alpha$ as $\alpha\to 0$
\begin{enumerate}
    \item  If $K$ is injective and $r\in \calr(K)$ then
          $\norm{\varphi_\alpha-\varphi_0}\to 0$. Define bias as
          $\varphi_\alpha-\varphi$, we have \begin{equation*}
              \begin{split}
                  b_\alpha&=(\alpha I+K^*K)^{-1}K^*Kr-\varphi\\
                  &=(\alpha I+K^*K)^{-1}(K^*K-\alpha I+K^*K)\varphi\\
                  &=-\alpha(\alpha I+K^*K)^{-1}\varphi
              \end{split}
          \end{equation*}
          Therefore as in the derivation of \eqref{eq:varphi_alpha}, $\norm{b_\alpha} = \alpha^2\sum \frac{1}{\lambda_i^2+\alpha^2}\angs{\varphi,\varphi_i}^2\to 0$
    \item If $K$ is not injective, then $\varphi=\varphi_1+\varphi_0$ where $\varphi_1\in
              \caln(K)$ then $\norm{\varphi_\alpha-\varphi_1}\to 0$, not the true $\varphi$.
\end{enumerate}

\subsubsection{Sauce condition}
We introduce a sauce condition to control the rate of convergence of
$\varphi_\alpha$. The condition that $\varphi\in\calr(K^*K)^{\frac{\beta}{2}}$
is equivalent to there exists a $\delta\in (K^*K)^{\frac{\beta}{2}}\cale$ such
that $\varphi=K^*K\delta$. If $\varphi$ satisfies the sauce condition, then by
SVD of $K$ we have \begin{equation*}
    \varphi=\sum_i \lambda_i^\beta\angs{\delta,\varphi_i}\varphi_i
\end{equation*} {\color{red} (why?)}
In general we have $\varphi=\sum_i \angs{\varphi,\varphi_i}\varphi_i$.
Thus, \begin{align*}
    \angs{\varphi,\varphi_i} & =\lambda_i^\beta\angs{\delta,\varphi_i} \\ \Rightarrow \sum_i\frac{\angs{\varphi,\varphi_i}^2}{\lambda_j^{2\beta}}&=\angs{\delta,\varphi_j}^2 < \infty
\end{align*}
Under this condition, we can show that $\norm{b_\alpha}^2 = O(\alpha^{\beta})$ for $\beta<2$ ($O (\alpha^{\beta\wedge 2})$).