\fancyhead[R]{Lecture 3}
\subsection{Linear operator}
\begin{definition}[linear operator]
    Let $\cale$ and $\calh$ be two Hilbert spaces (equipped with scalar product). A linear operator $K:\cale\rightarrow\calh$ is a function such that
    \begin{equation*}
        K\pa{\alpha_1\phi_1+\alpha_2\phi_2}=\alpha_1K\phi_1+\alpha_2K\phi_2
    \end{equation*}
    for all $\phi_1,\phi_2\in\cale$ and $\alpha_1,\alpha_2\in\R$.
\end{definition}
\begin{example}
    Let $X,Y,Z$ be three random variables defined on the $\pspace$. Let $X=Y\times Z$. We construct three $L^2$ spaces $L_X^2,L_Y^2,L_Z^2$. Define a linear operator $K$ such that $K\phi= \E\pa{\phi(y)\mid Z=z}$. Then $K$ is a linear operator from $L_Y^2$ to $L_Z^2$.
\end{example}
For a linear operator, we define the corresponding subspaces defined as domain, range and kernel as follows:
\begin{definition}[domain, range and kernel]
    Let $\cale$ and $\calh$ be two Hilbert spaces and $K:\cale\rightarrow\calh$ be a linear operator. Then
    \begin{itemize}
        \item The domain of $K$ is defined as
              \begin{equation*}
                  \cald\pa{K}=\setbra{\phi\in\cale:K\phi\in\calh}
              \end{equation*}
        \item The range of $K$ is defined as
              \begin{equation*}
                  \calr\pa{K}=\setbra{K\phi:\phi\in\calm\pa{K}}
              \end{equation*}
        \item The kernel of $K$ is defined as
              \begin{equation*}
                  \caln\pa{K}=\setbra{\phi\in\calm\pa{K}:K\phi=0}
              \end{equation*}
    \end{itemize}
\end{definition}

For completeness, we define injection and surjection as follows:
\begin{definition}[injection and surjection]
    Let $\cale$ and $\calh$ be two Hilbert spaces and $K:\cale\rightarrow\calh$ be a linear operator. Then
    \begin{itemize}
        \item $K$ is called an injection if for all $\phi_1,\phi_2\in\cale$, $K\phi_1=K\phi_2$ implies $\phi_1=\phi_2$.
        \item $K$ is called a surjection if for all $h\in\calh$, there exists $\phi\in\cale$ such that $K\phi=h$.
    \end{itemize}
\end{definition}

\subsubsection{Boundness and continuity}
\begin{definition}[boundness]
    Let $\cale$ and $\calh$ be two Hilbert spaces and $K:\cale\rightarrow\calh$ be a linear operator. Then $K$ is called bounded if there exists a constant $C>0$ such that
    \begin{equation*}
        \norm{K\phi}_{\calh}\leq C\norm{\phi}_{\cale}
    \end{equation*}
    for all $\phi\in\cale$.
\end{definition}
\begin{definition}[continuity]
    Let $\cale$ and $\calh$ be two Hilbert spaces and $K:\cale\rightarrow\calh$ be a linear operator. Then $K$ is called continuous if for all $\phi_n\rightarrow\phi$ in $\cale$, we have $K\phi_n\rightarrow K\phi$ in $\calh$.
\end{definition}
We will look at an example where $k$ is not continuous.
\begin{example}
    Let $\cale=\calc_{\bra{0,1}}^0$ be the space of continuous functions on $\bra{0,1}$. Define a linear operator $K:\cale\rightarrow\R$ such that $K\phi=\phi\pa{x_0}$.
\end{example}

\subsubsection{Adjoint operator} Similar to the transpose of a matrix, we define the adjoint operator of a
linear operator.
\begin{definition}[adjoint operator]
    Let $\cale$ and $\calh$ be two Hilbert spaces and $K:\cale\rightarrow\calh$ be a linear operator. Then the adjoint operator $K^*:\calh\rightarrow\cale$ is defined as
    \begin{equation*}
        \angs{K\phi,h}_{\calh}=\angs{\phi,K^*h}_{\cale}
    \end{equation*}
    for all $\phi\in\cale$ and $h\in\calh$.
\end{definition}
\begin{remark}
    We can show that matrix transpose is a special case of adjoint operator. For example, let $A$ be a matrix of dimension $m\times n$ and $x,y$ be vectors of dimension $n,m$ respectively. Then we have $\angs{Ax,y}=(Ax)^\top y= x^\top A^\top y = \angs{x,A^*y}$.
\end{remark}
\begin{example}[integral operator as a self-adjoint operator]
    Let $\cale = \calc^0\pa{\bra{0,1}}=\F$. Define a linear operator $K:\cale\rightarrow\F$ such that $K\phi\pa{x}=\int_0^1\phi\pa{x}k\pa{x,y}dx= \psi\pa{y}$ where $k\pa{x,y}$ is a given function. Then for any $\psi\in \F$, we have \begin{equation*}
        \begin{split}
            \angs{K\phi,\psi}_{\F}&=\int_0^1\int_0^1\phi\pa{x}k\pa{x,y}dx\psi(y)dy\\&=\int_0^1\int_0^1k\pa{x,y}\phi\pa{x}\psi(y)dxdy\\&=\angs{\phi,K^*\psi}_{\cale}
        \end{split}
    \end{equation*} Then we have $K^*\psi\pa{y}=\int_0^1k\pa{x,y}\psi\pa{y}dy$. We say $K$ is \textbf{self-adjoint} if $k\pa{x,y}=k\pa{y,x}$, that is $K^*=K$.
\end{example}
We look at another example in the context of probability.
\begin{example}
    Let $X=Y \times Z$, where $X,Y,Z$ are random variables. We define a linear operator $K$ from $L_Y^2 \to L_Z^2$ such that $K\phi=\E\pa{\phi\pa{Y}\mid Z=z}$. Then the adjoint operator $K^*$ is given by $K^*h\pa{y}=\int h\pa{z}f\pa{z\mid y}dz$. Because by definition,
    \begin{equation*}
        \begin{split}
            \angs{K\phi,\psi}&=\angs{\E\pa{\phi\pa{Y}|Z},\psi(Z)}\\
            & = \E\bra{\E\pa{\phi\pa{Y}| Z}\psi(Z)}\\
            & = \E\bra{\phi\pa{Y}\psi(Z)}\quad \text{by independence?}\\
            & = \E\bra{\phi\pa{Y}\E\pa{\psi(Z)| Y}}\\
            & = \angs{\phi,\E\pa{\psi(Z)| Y}}
        \end{split}
    \end{equation*}
\end{example}

\subsubsection{Compact operators and SVD}
\begin{definition}[Singular value decompostion]
    Let $\cale$ and $\F$ be two Hilbert spaces and $K:\cale\rightarrow\F$ be a linear compact operator. The adjoint operator $K^*$ is also compact. Then there exists a set of
    \begin{enumerate}
        \item A set of singular values $\lambda_j\geq 0$
        \item Two orthonormal basis $\phi_j\in\cale$ and $\psi_j\in\F$ such that $\forall
                  \phi\in\cale$, we have $\phi=\sum_j\lambda_j\angs{\phi,\phi_j}\phi_j+\phi_0$
              for $\phi_0\in \caln\pa{K}$. Similarly for $\psi\in \F$.
    \end{enumerate}
    The following properties hold from SVD:
    \begin{itemize}
        \item $K^*K\phi_j=\lambda_j^2\phi_j$ and $KK^*\psi_j=\lambda_j^2\psi_j$
        \item $K\phi_i=\lambda_j\psi_j$ and $K^*\psi_j=\lambda_j\phi_i$
        \item Main implication: \begin{equation*}
                  \begin{split}
                      K\phi&=\sum_j\angs{\phi,\phi_j}k\phi_j+K\phi_0\\
                      & = \sum_j\angs{\phi,\phi_j}\lambda_j\psi_j+0\\
                      & = \sum_j\lambda_j\angs{\phi,\phi_j}\psi_j
                  \end{split}
              \end{equation*}
    \end{itemize}
    which leads to the following decomposition:
\end{definition}

\begin{definition}[Hilbert Schimidt and Nuclear operator]
    A compact operator $K$ is called a Hilbert Schimidt operator if the sum of the squares of the singular values is finite, that is $\sum_j\lambda_j^2<\infty$.
    It is a nuclear operator if the sum of the singular values is finite, that is $\sum_j\lambda_j<\infty$.
\end{definition}
A direct consequence is that \begin{equation*}
    K^\beta\phi=\sum_j\lambda_j^\beta\angs{\phi,\phi_j}\psi_j
\end{equation*}

\begin{remark}
    We have the following inclusion property of the operator \begin{equation*}
        \text{Nuclear} \subset \text{Hilbert Schimidt} \subset \text{Compact} \subset \text{Continuous and Bounded} \subset \text{Linear}
    \end{equation*}
\end{remark}

\begin{theorem}
    Let $K$ be a compact operator from the Hilbert space $\cale$ to $\F$. Denote any orthonormal basis of $\cale$ by $\setbra{\varepsilon_j}_{j\in J}$. Then the following properties hold:
    \begin{itemize}
        \item $\sum \lambda_j^2 = \sum_j \angs{K\varepsilon_j,K\varepsilon_j}=\sum_j \angs{\varepsilon_j,K^*K\varepsilon_j}$
        \item $\sum \lambda_j = \sum_j \angs{K\varepsilon_j,\varepsilon_j}$
    \end{itemize}
\end{theorem}
The theorem is obviously true if $\set{\varepsilon_j}$ is the basis from the SVD of $K$.

% \subsection{Linear equations}
% \begin{definition}[generalized inverse]
%     $$ \min \norm{k\phi-r}^2 $$ has a solution if and only if $r\in\calr\pa{K}+\calr\pa{k}^\perp$.

% \end{definition}

% \section{}
% Given the linear equation $$r=K\phi$$ for $K: \cale \to \calh$ in Hilbert spaces.
% $K$ is compact and injective.
% The regularized soltuion $$\phi_a= (K^*K+\alpha I)^{-1}K^*r=\sum \frac{\lambda}{}$$

% 2 is qualification of Tikhonov regularization condition.

% In \cale, L is a differential operator. It is defined on a dense subset on $\cale$ (not defined everywhere).
% $K$ is an operator from $\cale$ to $\calh$. $L^{-a}$ is an integral operator. We say $K\equiv L^{-a}$ because

% The degree of illposeness of $K$ w.r.t $L$ is $a$.
% $L^b \phi $ defined 

% Compare K to L and \phi to L. Then

% \begin{example}[Spectral cut-off]
%     \phi=\sum_j \frac{1}{\lambda_j} \angs{r,\psi_j}\phi_j
%     \phi_\epsilon=\sum_{j/lambda_j>\epsilon} \frac{1}{\lambda_j} \angs{r,\psi_j}\phi_j

%     \norm{\phi_\epsilon-\phi}^2=\sum_{j/lambda_j<\epsilon} 
% \end{example}
